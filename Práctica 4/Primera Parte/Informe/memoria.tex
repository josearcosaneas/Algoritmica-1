\input{preambuloSimple.tex}
 \usepackage{algpseudocode}
%----------------------------------------------------------------------------------------
%	TÍTULO Y DATOS DEL ALUMNO
%----------------------------------------------------------------------------------------

\title{	
\normalfont \normalsize 
\textsc{{\bf Algorítmica (2015-2016)} \\ Grado en Ingeniería Informática \\ Universidad de Granada} \\ [25pt] % Your university, school and/or department name(s)
\horrule{0.5pt} \\[0.4cm] % Thin top horizontal rule
\huge Práctica 4-Primera Parte: Cena de gala \\ % The assignment title
\horrule{2pt} \\[0.5cm] % Thick bottom horizontal rule
}

\author{Francisco Carrillo Pérez,Borja Cañavate Bordons, \\Miguel Porcel Jiménez,Jose Manuel Rejón Santiago,Jose Arcos Aneas} % Nombre y apellidos

\date{\normalsize\today} % Incluye la fecha actual

%----------------------------------------------------------------------------------------
% DOCUMENTO
%----------------------------------------------------------------------------------------

\begin{document}

\maketitle % Muestra el Título

\newpage %inserta un salto de página

\tableofcontents % para generar el índice de contenidos

\listoffigures

\listoftables

\newpage

\section{Introducción }

	El objetivo de esta práctica es diseñar un algoritmo Bactraking, que resuelva uno de los cinco problemas de la práctica y realizar un estudio empírico de su eficiencia.
	
	Se desea sentar a N invitados alrededor de una mesa, de manera que cada invitado tendra a su lado a otros dos. Cada par de invitados tiene un nivel de compatibilidad. Se desea maximizar la compatibilidad de estos comensales.
%----------------------------------------------------------------------------------------
\section{Resolución del problema}

Se elige un elemento como el raiz.
Se mete en el vector de solución y se elimina de los posibles candidatos.
Se generan todos los posibles hijos, estos serán todos los que no se hayan incluido ya en el vector solución.
Si la longitud de nuestra solución igual al numero de comensales es solución.\\


Backtracking(A[], k)\\
		Si Es\_Solucion(A[], k)\\
			Entonces Procesar\_Solucion(A[], k)\\
		SinoPor cada Encontrar\_Sucesores(A[], k) hacer\\
			Backtracking(A[], k+1)\\
	

%------------------------------------------------------------------------------------------

\section{Elementos de la solución al problema}

\subsection{Representación de la compatibilidad}
La entrada sera una matriz simetrica de valores aleatorios con la diagonal de 0s.

\subsection{Representación de la solución}
Vector de longitud igual al número de invitados (\textit{N}), en que cada posición guarda el valor del invitado que se sienta en la posición \textit{i}.

\subsection{Solucion parcial}
olucion parcial al problema de tamaño menor que N.

\subsection{Función de poda}
No se me ocurre nada.

\subsection{Restricciones explícitas}
Los valores que puede tomar la solucion son los enteros de 1 a N. Donde N es el número total de invitados. 

\subsection{Restricciones implícitas}
Estas restricciones son las que determinan si una función parcial puede llevarnos a una solucion del problema. Si supera un umbral. 

%--------------------------------------------------------------------------------------------

\section{Pseudocódigo}


\end{document}