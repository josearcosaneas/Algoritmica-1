\input{preambuloSimple.tex}
 \usepackage{algpseudocode}
%----------------------------------------------------------------------------------------
%	TÍTULO Y DATOS DEL ALUMNO
%----------------------------------------------------------------------------------------

\title{	
\normalfont \normalsize 
\textsc{{\bf Algorítmica (2015-2016)} \\ Grado en Ingeniería Informática \\ Universidad de Granada} \\ [25pt] % Your university, school and/or department name(s)
\horrule{0.5pt} \\[0.4cm] % Thin top horizontal rule
\huge Práctica 3-Primera Parte: Minimizando el número de visitas al proveedor \\ % The assignment title
\horrule{2pt} \\[0.5cm] % Thick bottom horizontal rule
}

\author{Francisco Carrillo Pérez,Borja Cañavate Bordons, \\Miguel Porcel Jiménez,Jose Manuel Rejón Santiago,Jose Arcos Aneas} % Nombre y apellidos

\date{\normalsize\today} % Incluye la fecha actual

%----------------------------------------------------------------------------------------
% DOCUMENTO
%----------------------------------------------------------------------------------------

\begin{document}

\maketitle % Muestra el Título

\newpage %inserta un salto de página

\tableofcontents % para generar el índice de contenidos

\listoffigures

\listoftables

\newpage

\section{Introducción }

El problema que vamos a tratar de resolver desde la aproximación de un algoritmo Greedy consiste en minimizar el número de visitas que debe realizar un granjero a su proveedor, es decir, nos encontramos frente a un problema de optimización.

La máquina que hemos utilizado tiene las siguientes características:
	
\begin{itemize}
		
	\item Procesador: Intel Core i5-3337U (2.7GHz x 2)
	\item Memoria RAM: 4GB
	\item Disco Duro: 500GB 5400 rpm
	\item SO: Manjaro Linux 15.2 Capella 64 bits
\end{itemize}
	
%----------------------------------------------------------------------------------------

\section{Resolución del problema}

Tenemos un granjero cuyo fertilizante dura R días. Además conocemos los días en los que abre la tienda.
Con estas características podemos tener dos posibilidades para resolver este problema:
\begin{itemize}
	\item Buscar la fecha mas cercana dentro del intervalo R
	\item Buscar la fecha mas lejana dentro del intervalor R
\end{itemize}

Como lo que queremos es optimizar, es decir, reducir el número de viajes que debe dar el granjero descartamos la primera opción y nos centramos en la segunda.

%------------------------------------------------------------------------------------------

\section{Elementos de la solución al problema}

\subsection{Conjunto de candidatos}
El conjunto de candiatos serán aquellos días en los que la tienda se encuentre abierta.

\subsection{Conjunto de seleccionados}
El conjunto de seleccionados serán los días elegidos para acudir a la tienda.

\subsection{Función solución}
Nuestra función solución será si han pasado todos esos días y no nos hemos quedado sin fertilizante, lo que significará que hemos maximizado el número de viajes.

\subsection{Función de factibilidad}
Sabremos que no hay solución, si dentro del intervalo R no hay ninguna tienda a la que acudir.

\subsection{Función selección}
Se seleccionará el día más lejano de los posibles.

\subsection{Función objetivo}
La solución devolverá una lista, con los dias en los que el granjero irá a  la tienda.

%--------------------------------------------------------------------------------------------

\section{Pseudocódigo}



\end{document}